\documentclass[12pt]{article}
\usepackage[english]{babel}
\usepackage[utf8]{inputenc}
\title{Golem Network \\ Net Module Notes}
\begin{document}
\maketitle{}
\section{Introduction}

This documment describes how currently component for node communication works 
and how it shud work in the future.

\section{General net abstraction}

% TODO: Opisać generalna abstrakcję rozproszonego RPC które musi 
%   zaimplementować taki komponent.

\subsection{Level0 Implementation}

is called "CentralNet".

\subsubsection{Assumtions}

\begin{itemize}
    \item Easy to implement.
    \item Musi umieć przebijąc firewall'a. 
    \item Nie musi byś wydajne
    \item Nie musi być zdecentaralizowane.
    \item Nie musi wspirać szyfrowania e2e. 
\end{itemize}

Cel: Zbudowanie PoC dla użytkowników.

Wszystkie węzły łączą się do jednego centralnego serwera. Po TCP. Komummikacja jest klienta serwer.


Protokół wygląda tak:

 - Klient łaczy się od serwera
 - Klient wysyła hello ( z wersją bibliotek). -> hello.
 - Klient wysyła komende register (po to by móc odbieraz komunikaty wysyłanie do niego)
 - Serwer odpowiada 
 - Serwer co jakiś czas wysyała ping / na który klient odpowiada pong.


 - jak trzeba wysłać do jakiego węzła to pakije jest w komunikat CallRequest

 \begin{code}
    message CallRequest {
        string caller = 1;
        string address = 2;
        string request_id = 3;
        bytes data = 4;
        bool no_reply = 5;
      }
 \end{code}


Komunikacja do proxy uzywa TLS1.3. Brak szyfrowania e2e.


\subsection{Market a CentralNet}

Market wymaga rozsyłanie informacji do wielu węzłów.
W tym celu używa się funkcjonalności bradcast implamentacji CentralNet'u.

Węzły kótee szujkją ofert rejestruje się jako odbiorcy na kanale rozgłaszania.
Kazdy providera co pewien czas (domyślnie 2min) wysyła na ten kanał inofrmację 
o tym jakie posiada oferty. dzieki temy requestor które szuka ofert po jakimś czasie poznaje 
wszystkie ofert providerów podłaczonych do danego węzła CentralNet.


\subsection{Horyzontalne skalowanie CentralNet (może lepeij o tym nie pisać)}

Głównym ograniczeniem jest mechanizm broadcastów. możliwe są dwie konfogiracje:

1. Węzły łaczą się tylko do jednej losowej instancji serwer (stabilinie wg. id identyfikatora węzła) 
w takiej sytuacji węzłe requestora widzie tylko częśc providerów ta która jest podpieta do tego samego proxy.

2. Węzłę który bedzie dużo zlecać komfigurujemy by łaczył się do wielu serwerów proxy. spowoduje to że będzie dostałwał 
odpowiednio wiecej komunikatów o ofertach.


\section{Level1 Hybrid net}



\section{LeveL2 WebRTC}

- rfc5389 Session Traversal Utilities for NAT (STUN) 
- rfc5766 Traversal Using Relays around NAT (TURN)
- rfc8445 Interactive Connectivity Establishment (ICE)






\end{document}
