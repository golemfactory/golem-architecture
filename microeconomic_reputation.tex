\documentclass{article}
\usepackage{amsfonts}
\usepackage{hyperref}

\title{Strategies and reputation - a microeconomic description of the Golem marketplace}
\author{Jan Betley}

\begin{document}
\maketitle
\section{Introduction}
This document provides a simple microeconomic model of the Golem marketplace. 
Following questions are answered within the model:

\begin{itemize}
\item What do we exactly mean by the "reputation"?
\item What is the exact problem we hope to solve with the reputation mechanism?
\item What are the other, non-reputation approaches to the main problem?
\item How to measure the quality of our solutions? I.e. how to know if we succeeded?
\end{itemize}


The problem is approached from the highest point of view - no specific solutions are proposed, only general classes of solutions are discussed.
With an exception of a few details/examples this document describes just a "general" marketplace (replace "provider/requestor" with "seller/buyer").

\section{Definitions}

\textbf{Agent} - A decision-making entity (person, company, etc.).
\newline
\textbf{Utility} - A total happiness of an agent.
\newline
\textbf{Utility function $U$} - A function $U: StateOfTheWorld -> \mathbb{R}$, defined for a particular agent, that represents the agents utility in a given situation.
\newline
\textbf{Expected utility} - For every possible decision an agent can make ($D$) and every possible state of the world $w$, there is some probability $P: (D, w) \rightarrow [0, 1]$  that given decision will lead to the given state of the world.
Agents don't have the full knowledge about $P$, they know only some information $I$ and an estimation of $P$ based on this information: $P_I: (D, w) \rightarrow [0, 1]$. The expected utility $E: (I, U, D) -> \mathbb{R}$ is defined as follows:

$$
E(I, U, D) = \sum_{w \in all\_possible\_world\_states}U(w) * P_I(D, w)
$$

We'll usually use a shorter notation $E(I, U, D) = E_A(D)$ to describe "expected utility of agent A, who has information $I$ and utility function $D$".

In other words, for agent $A$ and decision $D$, $E_A(D)$ is "how happy agent $A$ expects to be if they do $D$".
\newline

\textbf{Golem} - [TODO better definition?] Everything that influences the utility resulting from the interactions with the Golem Network protocol - the protocol itself, available software, state of the market etc.
\newline

\section{Additional assumptions}

\begin{enumerate}
\item The \href{https://en.wikipedia.org/wiki/Expected_utility_hypothesis}{expected utility hipotesis}: every agent tries to maximize their expected utility.

\item Every agent can costlessy access and analyze all of the publicly available information about Golem. Thanks to this assumption we can remove $I$ from the equations - we no longer
    care about "what agent knows", but only about "what really is". This is a major simplification that might not be a good approximation of the "real world" Golem market, but we
    accept it for the sake of the brevity of this document. Also "truth about Golem is known" is an ideal world we would like to live in.

\item The expected utility and money are interchangeable, that is - for every agent $A$ and a pair of decisions $(D_1, D_2): E_A(D_1) < E_A(D_2)$, there is an amount of money X that
    can be given to the agent so that $E_A(D_1) + U_A(X) = E_A(D_2)$. This way we can treat utilities as if they were money, and thus compare them between different agents.

\item Lets define the value of Golem as:

$$
        V_G = \sum_{A \in all\_agents\_using\_Golem}E_A(USE\_GOLEM)
$$
(Note that - because of assumptions 1 and 2 - $E_A$ is always positive, decision "use Golem" is only made by agents who profit from it).

We assume that maximization of $V_G$ is one of the goals of the Golem Factory - and from the point of view of this document, the only goal.
\end{enumerate}

\end{document}
