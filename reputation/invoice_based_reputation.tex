\documentclass{article}
\usepackage{amsfonts}
\usepackage{amsmath}
\usepackage{hyperref}


\begin{document}

\section{The main concept}
\subsection{Assumptions}

This document is a proposal of an implementation of the reputation model described in the "Microeconomic Golem Reputation"\footnote{
\href{https://github.com/golemfactory/golem-architecture/blob/jb/microeconomy-reputation/reputation/microeconomic\_reputation.pdf}
     {https://github.com/golemfactory/golem-architecture/blob/jb/microeconomy-reputation/reputation/microeconomic\_reputation.pdf}
[TODO - beter url when this is merged to some final location]}, and thus accepts all assumptions made there, most important being
the expected utility maximization.
    
Additional assumptions:

\begin{enumerate}
    \item The ground truth about all invoices on the market (amount, sender, receiver) is known.
    \item Obtaining \& parsing the data about invoices/payments costs neither money nor time.
    \item The only information provider cares about is "how much I will be paid?" and the only information requestor cares about
        is the quality of the services received.
    \item The market is balanced, i.e.: 
        \begin{enumerate}
            \item Good enough and sufficiently cheap provider will spend no more than X\% of time waiting for agreements, for a reasonable X (e.g. 10\%).
            \item Requestor willing to pay a sufficient amount of money will find enough providers fast enough
        \end{enumerate}
\end{enumerate}

Some of them are questionable even as approximations - there is a further section discussing them.

\subsection{General idea}

The full knowledge about invoices and payments gives us a crude approximation of the "reputation-like" information:
\begin{enumerate}
    \item Provider POV: the requestor who paid bigger part of their past invoices will more likely pay another invoice than another requestor
        who paid less past invoices.
    \item Requestor POV: the provider whose invoices are usually paid more likely provides high quality services than a provider who is paid
        only rarely or never.
\end{enumerate}

Justification behind these statements:
\begin{itemize}
    \item Both first and second: agent's strategy rarely changes, so it's likely that when trading with us
        they will behave similarly to the way they behaved in the similiar past situations.
    \item Second: if a requestor doesn't pay the provider, they are less likely to trade with them in the future (e.g. because of a provider's local history), 
        and thus requestors more often pay providers they want to trade with, and those are more likely the providers we want to trade with.
\end{itemize}

\subsection{Detailed strategies}

Provider strategy:
\begin{enumerate}
    \item Gather a local history of all our contracts - (requestor\_id, invoice\_amount, paid\_amount) + additional details (e.g. agreement length, initial demand etc).
    \item Send offers to the market with "base" prices, that will later (in the negotiation phase) be increased for untrusted requestors.
    \item When a requestor responds to initial our offer:
        \begin{itemize}
            \item If this is a new requestor (i.e. one that didn't yet trade with anyone on the market, or didn't trade enough):
                \begin{enumerate}
                    \item Counteroffer with a "payment in advance" property
                    \item If accepted, start with sending a debit note (for some fixed amount) with a short payment time
                    \item Refuse to start an activity until the first debit note is paid
                    \item Continue in the same manner, i.e. request upfront payments
                \end{enumerate}
            \item If this is a requestor we've already traded with, and they didn't pay all the due amount:
                \begin{enumerate}
                    \item Add the missing amount (possibly with some discounting factor) to the fixed amount
                    \item If agreed, start with sending a debit note for the missing amount, with short payment time
                    \item Refuse to start an activity until the debit note is paid
                \end{enumerate}
            \item In other cases, i.e. when requestor has some experience on the market and they either didn't trade with us or paid a full amount:
                \begin{enumerate}
                    \item Estimate the \% of the final invoice we expect the requestor to pay us. In the most primitive version this might be just the \%
                        of their invoices they paid until now, although this can (and should) be done much better\footnote{E.g. if they didn't pay a provider that is
                        never paid, there's a big chance this is an invalid provider who just does no work at all, but if they didn't pay a provider who's usually
                        paid fully then probably something's wrong with the requestora.
                        We should also include in the estimations few other types of the information, e.g.
                        \begin{itemize}
                            \item Our own experiences with this requestor (i.e. requestor who dutifully paid us a lot will probably pay again).
                            \item Expected agreement length, proposed payment interval etc.
                        \end{itemize}}
                    \item If we estimate $\sim$100\% of the invoice will be paid, accept the offer. 
                    \item If we estimate less will be paid, reply with a counter-offer with higher prices, so that they compensate for the \% of the invoice
                        we expect not to be paid.
                \end{enumerate}
        \end{itemize}
\end{enumerate}

Requestor strategy:
[TODO]

\subsection{The reputation deployment process}

So we have defined some strategies for requestors/providers.
Who will use them? How to encourage others to use similar strategies?


\section{Discussion}
\subsection{Assumption 1 - ground truth about invoices}
\subsection{Assumption 2 - obtaining/parsing invoice data}
\subsection{Assumption 3 - agents' utility functions}
\subsection{Assumption 4 - market balance}

\end{document}
