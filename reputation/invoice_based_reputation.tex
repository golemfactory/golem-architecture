\documentclass{article}
\usepackage{amsfonts}
\usepackage{amsmath}
\usepackage{hyperref}

\title{Invoice - based reputation system}
\author{Jan Betley\\ Golem Factory}

\begin{document}
\maketitle

\section{The main concept}

General note: this is a crude description of an idea, not a "final" design I'm covinced will work.
IOW, this document should be enough for discussions and future plans, not something we can now go and implement.

\subsection{Assumptions}

The most important assumption in this document is the expected utility maximization hypothesis\footnote{
\href{https://en.wikipedia.org/wiki/Expected\_utility\_hypothesis}
     {https://en.wikipedia.org/wiki/Expected\_utility\_hypothesis}
}: providers and requestors will do what they think is best for them, and the main purpose of the reputation system
is to construct an environment where the most profitable behaviour is also the desired behaviour.
    
Few more assumptions (some of them are questionable even as approximations - there is a further section discussing them):

\begin{enumerate}
    \item The ground truth about all payments on the market (timestamp, amount, sender, receiver) is known.
    \item The ground truth about all invoices on the market (timestamp, amount, sender, receiver) is known.
    \item Obtaining \& parsing the data about invoices/payments costs neither money nor time.
    \item The only thing provider cares about is "how much I will be paid?" and the only thing requestor cares about
        are the quality of the services received and amount paid.
    \item The market is balanced, i.e.: 
        \begin{enumerate}
            \item Good enough and sufficiently cheap provider will spend no more than X\% of time waiting for agreements, for a reasonable X (e.g. 10\%).
            \item Requestor willing to pay a sufficient amount of money will find enough providers fast enough
        \end{enumerate}
\end{enumerate}


\subsection{General idea}

The full knowledge about invoices and payments gives us a crude approximation of the "reputation-like" information: 
good requestors pay, good providers are paid. In more detail:
\begin{enumerate}
    \item Provider POV: the requestor who paid bigger part of their past invoices will more likely pay another invoice than another requestor
        who paid less past invoices.
    \item Requestor POV: the provider whose invoices are usually paid more likely provides high quality services than a provider who is paid
        only rarely or never.
\end{enumerate}

Justification behind these statements:
\begin{itemize}
    \item Both first and second: providers/requestors market strategy rarely changes, so it's likely that when trading with us
        they will behave similarly to the way they behaved in the similar past situations.
    \item Second: if a requestor doesn't pay the provider, they are less likely to trade with them in the future (e.g. because of the provider's local history), 
        and thus requestors more often pay providers they want to trade with, and those are more likely the providers we want to trade with.
\end{itemize}

\subsection{Detailed strategies}

There are countless possible strategies that are consistent with the main idea, strategies in this section are just examples.

Provider strategy:
\begin{enumerate}
    \item Gather a local history of all our contracts - (requestor id, invoice amount, paid amount) + additional details (e.g. agreement length, initial demand etc).
    \item Send offers to the market with "base" prices, that will later (in the negotiation phase) be increased for untrusted requestors.
    \item When a requestor responds to the our initial offer:
        \begin{itemize}
            \item If this is a new requestor (i.e. one that didn't yet trade with anyone on the market, or didn't trade enough):
                \begin{enumerate}
                    \item Counteroffer with a "payment in advance" property
                    \item If accepted, start with sending a debit note (for some fixed amount) with a short payment time
                    \item Refuse to start an activity until the first debit note is paid
                    \item Continue in the same manner, i.e. request upfront payments
                \end{enumerate}
            \item If this is a requestor we've already traded with, and they didn't pay all the due amount:
                \begin{enumerate}
                    \item Add the missing amount (possibly with some discounting factor) to the fixed amount
                    \item If agreed, start with sending a debit note for the missing amount, with short payment time
                    \item Refuse to start an activity until the debit note is paid
                \end{enumerate}
            \item In other cases, i.e. when requestor has some experience on the market and they either didn't trade with us or paid a full amount:
                \begin{enumerate}
                    \item Estimate the \% of the final invoice we expect the requestor to pay us. In the most primitive version this might be just the \%
                        of their invoices they paid until now, although this can (and should) be done much better\footnote{E.g. if they didn't pay a provider that is
                        never paid, there's a big chance this is an invalid provider who just does no work at all, but if they didn't pay a provider who's usually
                        paid fully then probably something's wrong with the requestora.
                        We should also include in the estimations few other types of the information, e.g.
                        \begin{itemize}
                            \item Our own experiences with this requestor (i.e. requestor who dutifully paid us a lot will probably pay again).
                            \item Expected agreement length, proposed payment interval etc.
                        \end{itemize}}
                    \item If we estimate $\sim$100\% of the invoice will be paid, accept the offer. 
                    \item If we estimate less will be paid, reply with a counter-offer with higher prices, so that they compensate for the \% of the invoice
                        we expect not to be paid.
                \end{enumerate}
        \end{itemize}
\end{enumerate}

Requestor strategy:
\begin{enumerate}
    \item Gather similar local history as the provider
    \item Sign agreements that require an advance payment only with providers you have a reason to trust 
        (e.g. because you traded with them before or because they have a good history on the market)
    \item Except for the previous point, just try to sign the most profitable agreements, including any information about the "provider quality" you have.
        This should include "high quality providers are paid more often than low quality providers", but also other things like "provider benchmarks" or
        the history of the requestor's trades.
    \item Provider sends debit notes/invoices stating the expected amounts, but requestor decides freely about the amount paid (they might also decide not to pay at all).
        When deciding about the amount to pay, consider a following tradeof. The less money I pay:
        \begin{itemize}
            \item (gain) The more money I still own.
            \item (loss) The more expensive deals I will have with other providers, especially if I decide not to pay a provider who's usually paid.
            \item (loss) The more I will have to pay to trade with this provider again.
        \end{itemize}
        And pay any amount you consider best from your POV.\footnote{
            This should somehow converge to "the worse is the provider, the less we want to pay" - and that's exactly what we want.
        }
\end{enumerate}

\subsection{The reputation system deployment process}

So we have defined some strategies for requestors/providers.
Who will use them? How to encourage others to use similar strategies?
The plan looks as follows:
\begin{enumerate}
    \item Implement provider/requestor strategies described above. Set them as default provider strategies and recommended requestor strategies.
    \item Put on the market enough providers/requestors who use these strategies.
    \item (Crucial assumption that should be analyzed) When a new provider/requestor enters the market, they use these (or similar) strategies, because they
        are the most profitable. Thus, as the market grows, the adoption level of these strategies remains high.
    \item Finally, we can remove the initial providers/requestors and the reputation system will still work.
\end{enumerate}

\section{Discussion}
\subsection{Assumption 2 - ground truth about invoices}

This is hard, and this must be solved well for this whole idea to work.
There are two general directions:
\begin{enumerate}
    \item Implement some logic that will enforce this ground truth, e.g.:
        \begin{itemize}
            \item Put invoices on some additional chain
            \item Require invoices to be signed by both sides (?)
        \end{itemize}
    \item Drop this assumption, but modify the strategies so that "agreeing on the invoice amount" will be profitable for both sides, e.g.
        \begin{itemize}
            \item Both provider and requestor "report" invoices somewhere
            \item For the requestor it's always (or usually?) better to report the same value that was reported by the provider
            \item For the provider it's always (or usually?) better to have the invoice value confirmed by the requestor
        \end{itemize}
\end{enumerate}

\subsection{Assumption 3 - obtaining/parsing invoice data cost}

We'll need a node information API (GAP-14) that will gather the data and provide useful simple aggregates that will be consumed in strategies.

This looks like a fully centralized solution, but:
\begin{itemize}
    \item If the invoices are gathered in a public location (e.g. a side chain) then there can be multiple different APIs aggregating the same information.
    \item If the invoices are gathered in a database without a public access, then there could be some other entity who also gathers invoice data.
\end{itemize}

\subsection{Other assumptions}

\begin{itemize}

    \item Assumption 1 (payments ground truth) is true, because we pay in crypto.
    \item Assumption 4 (providers care only about money, requestors care only about service quality) is false\footnote{e.g. requestor would also want their data not to be stolen.}, but this is simply what this reputation system gives us - and it's quite a lot,
hopefully enough for a "serious" start of the market.
    \item Assumption 5 is fulfilled for any functioning market.
\end{itemize}

\subsection{Misc random notes}
\begin{enumerate}
    \item There is a malicious strategy available:
        \begin{enumerate}
            \item Put my own providers and requestors on the market
            \item Trade between them only until their reputation grows enough
            \item Use requestors to steal from other providers, use providers to steal from other requestors
        \end{enumerate}
        This is an important problem, but hopefully solvable:
        \begin{itemize}
            \item Because of the transaction costs, building up reputation this way costs
            \item Provider-requestor clusters like this will be visible in the data. E.g. a provider who got paid by X different requestors surely looks better than
                a provider who was paid multiple times by only one requestor.
        \end{itemize}
    \item Maybe debit notes would be better than invoices? This seems to be a good direction, because it gives us an additional useful information - 
        the amount accepted by the requestor. Accepting and not paying later is much worse than just rejecting the debit note.
    \item The obvious vulnerability is that if an estabilished requestor (i.e. not paying in advance) intends to leave the market, then there's no
        incentive for them to pay the last invoice(s). This is true, but not really important - estabilished requestors will be rarely deciding to leave the market.
\end{enumerate}
\end{document}
