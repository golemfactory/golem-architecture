\documentclass{article}
\usepackage{amsfonts}
\usepackage{amsmath}
\usepackage{hyperref}


\begin{document}

\section{The main concept}
\subsection{Assumptions}

This document is a proposal of an implementation of the reputation model described in the "Microeconomic Golem Reputation"\footnote{
\href{https://github.com/golemfactory/golem-architecture/blob/jb/microeconomy-reputation/reputation/microeconomic\_reputation.pdf}
     {https://github.com/golemfactory/golem-architecture/blob/jb/microeconomy-reputation/reputation/microeconomic\_reputation.pdf}
[TODO - beter url when this is merged to some final location]}, and thus accepts all assumptions made there, most important being
the expected utility maximization.
    
Additional assumptions:

\begin{enumerate}
    \item The ground truth about all invoices on the market (amount, sender, receiver) is known.
    \item Obtaining \& parsing the data about invoices/payments costs no money and/or time.
    \item The only information provider cares about is "how much I will be paid?" and the only information requestor cares about
        is the quality of the services received.
\end{enumerate}

They are questionable even as approximations, we'll weaken them for the more "real-life" scenario in the further part of the document.

\subsection{General idea}

The full knowledge about invoices and payments gives us a crude approximation of the "reputation-like" information:
\begin{enumerate}
    \item Provider POV: the requestor who paid bigger part of their past invoices will more likely pay another invoice than another requestor
        who paid less past invoices.
    \item Requestor POV: the provider whose invoices are usually paid more likely provides high quality services than a provider who is paid
        only rarely or never.
\end{enumerate}

Justification behind these statements:
\begin{itemize}
    \item Both first and second: agent's strategy rarely changes, so it's likely that when trading with us
        they will behave similarly to the way they behaved in the similiar past situations.
    \item Secod: if a requestor doesn't pay the provider, they are less likely to trade with them in the future (e.g. because of a provider's local history), 
        and thus requestors more often pay providers they want to trade with, and those are more likely the providers we want to trade with.
\end{itemize}

\subsection{Detailed strategies}

Provider strategy:
\begin{enumerate}
    \item Gather a local history, in a basic form (requestor\_id, amount\_due).

\end{document}
