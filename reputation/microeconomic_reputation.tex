\documentclass{article}
\usepackage{amsfonts}
\usepackage{amsmath}
\usepackage{hyperref}

\newtheorem{conclusion}{Conclusion}[section]

\title{Strategies and reputation - a microeconomic description of the Golem marketplace}
\author{Jan Betley}

\begin{document}
\maketitle
\section{Introduction}
\subsection{Abstract}
This document provides a simple microeconomic model of the Golem marketplace. 
Following questions are answered within the model:

\begin{itemize}
\item What do we exactly mean by the "reputation"?
\item What is the exact problem we hope to solve with the reputation mechanism?
\item What are the other, non-reputation approaches to the main problem?
\item How to measure the quality of our solutions? I.e. how to know if we succeeded?
\end{itemize}


The problem is approached from the highest point of view - no specific solutions are proposed, only general classes of solutions are discussed.
With an exception of a few details/examples this document describes just a "general" marketplace (replace "provider/requestor" with "seller/buyer").

\subsection{Definitions}

\textbf{Agent} - A decision-making entity (person, company, etc.).
\newline
\textbf{Utility} - A total happiness of an agent.
\newline
\textbf{Utility function $U$} - A function $U: StateOfTheWorld -> \mathbb{R}$, defined for a particular agent, that represents the agents utility in a given situation.
\newline
\textbf{Expected utility} - For every possible decision an agent can make ($D$) and every possible state of the world $w$, there is some probability $P: (D, w) \rightarrow [0, 1]$  that given decision will lead to the given state of the world.
Agents don't have the full knowledge about $P$, they know only some information $I$ and an estimation of $P$ based on this information: $P_I: (D, w) \rightarrow [0, 1]$. The expected utility $E: (I, U, D) -> \mathbb{R}$ is defined as follows:

\begin{equation}
E(I, U, D) = \sum_{w \in all\_possible\_world\_states}U(w) * P_I(D, w)
\end{equation}

We'll usually use a shorter notation $E(I, U, D) = E_A(D)$ to describe "expected utility of agent A, who has information $I$ and utility function $D$".

In other words, for agent $A$ and decision $D$, $E_A(D)$ is "how happy agent $A$ expects to be if they do $D$".
\newline

\textbf{Golem} - [TODO better definition?] Everything that influences the utility resulting from the interactions with the Golem Network protocol - the protocol itself, available software, state of the market etc.
\newline

\subsection{Additional assumptions}

\begin{enumerate}
\item The Expected Utility Hypothesis\footnote{\href{https://en.wikipedia.org/wiki/Expected\_utility\_hypothesis}{https://en.wikipedia.org/wiki/Expected\_utility\_hypothesis}}: every agent tries to maximize their expected utility.

\item Every agent can costlessy access and analyze all of the publicly available information about Golem. Thanks to this assumption we can remove $I$ from the equations - we no longer
    care about "what agent knows", but only about "what really is". This is a major simplification that might not be a good approximation of the "real world" Golem market, but we
    accept it for the sake of the brevity of this document. Also "truth about Golem is known" is an ideal world we would like to live in.

\item The expected utility and money are interchangeable, that is - for every agent $A$ and a pair of decisions $(D_1, D_2): E_A(D_1) < E_A(D_2)$, there is an amount of money X that
    can be given to the agent so that $E_A(D_1) + U_A(X) = E_A(D_2)$. This way we can treat utilities as if they were money, and thus compare them between different agents.

\item Lets define the Golem Value as:

\begin{equation}
    V_G = \sum_{A \in all\_agents\_using\_Golem}E_A(USE\_GOLEM)
\end{equation}
(Note that - because of assumptions 1 and 2 - $E_A$ is always positive, decision "use Golem" is only made by agents who profit from it).

We assume that maximization of $V_G$ is one of the goals of the Golem Factory - and from the point of view of this document, the only goal.
\end{enumerate}

\section{The purpose of the reputation}
\subsection{Dissolving the Golem Value}

Let's define $E_P(a)$ and $E_R(a)$ as the expected utility of action "take part in agreement $a$ as (respectively) provider/requestor", and $C_A$ as the cost of
the participation in the Golem market for agent $A$ that is not related to any particular agreement 
(e.g. opprotunity cost\footnote{\href{https://en.wikipedia.org/wiki/Opportunity\_cost}{https://en.wikipedia.org/wiki/Opportunity\_cost}} of the hardware used as a provider, 
or the cost of writing the requestor agent). Now, we can rewrite the $V_G$ equation as:

\begin{equation}
    V_G = \sum_{a \in all\_agreements}(E_P(a) + E_R(a)) - \sum_{A \in all\_agents\_using\_Golem}C_A
\end{equation}

Let's split $E_P(a)$ into few separate parts:

\begin{equation}
    E_P(a) = E_{PN}(a) + E_{PG}(a) - E_{PL}(a)
\end{equation}

Where $E_{PN}(a)$ is the nominal value of the agreement, $E_{PG}(a)$ is the providers additional gain that goes against the agreement 
(e.g. because of provider is using some other way the resources that should be available to the requestor), and $E_{PL}(a)$ are unexpected loses 
suffered by the provider because of the requestor not fulfilling their part of the agreement (e.g. by not paying the due amount).

$E_R(a)$ can be split in a similar way, and by putting this in the $V_G$ equation we get:

[TODO - define $E_N$, $E_G$ and $E_L$ and use them further]

\begin{equation}
\begin{split}
    V_G = \sum_{a \in all\_agreements}(E_{PN}(a) + E_{PG}(a) - E_{PL}(a) + E_{RN}(a) + E_{RG}(a) - E_{RL}(a)) \\
            - \sum_{A \in all\_agents\_using\_Golem}C_A \\
\end{split}
\end{equation}
\begin{equation}
\begin{split}
    V_G = \sum_{a \in all\_agreements}(E_{PN}(a) + E_{PG}(a)) \\
          - \sum_{a \in all\_agreements}(E_{PL}(a)- E_{RG}(a)) \\
          - \sum_{a \in all\_agreements}(E_{RL}(a)- E_{PG}(a)) \\
          - \sum_{A \in all\_agents\_using\_Golem}C_A
\end{split}
\end{equation}

Important note here is that both $\sum_{a \in all\_agreements}(E_{PL}(a)- E_{RG}(a))$ and $\sum_{a \in all\_agreements}(E_{PL}(a)- E_{RG}(a))$ are positive: 
when someone breaks the agreement, the harm done to the victim is usually greater then the offenders gain\footnote{TODO: examples/proof/justification?}.
Let's thus define the "expected value of the agreement $a$ value lost due to the participants not fulfilling their part of the agreement", 
$E_L(a) = (E_{PL}(a)- E_{RG}(a)) + (E_{RL}(a)- E_{PG}(a))$, and rewrite the equation one last time:

\begin{equation}
\begin{split}
    V_G = \sum_{a \in all\_agreements}(E_{PN}(a) + E_{RN}(a) - E_L(a)) \\
          - \sum_{A \in all\_agents\_using\_Golem}C_A
\end{split}
\end{equation}

This equation defines few general ways to increase the $V_G$:

\begin{enumerate}
\item Increase the number of agreements
\item Increase the average nominal value of the agreement (e.g. by providing additional capabilities, like internet connectivity or GPU access)
\item Decrease the average value lost because of actions against the agreement
\item Decrease the cost of the participation in the Golem market (e.g. by creating a better SDKs)
\end{enumerate}

Note that these ways interact with each other: e.g. if we implement some capabilities (to improve 2) that will be hard to use, we'll also worsen 4.
Or: the better is the average nominal value of the agreement, the more agreements we'll expect to have.
Or - what is important from the POV of this document - if we implement complex safeguards against cheating, they might have negative impact on all the other points.

Keeping that in mind, the rest of the document is aimed at the third direction from the above list.

\subsection{Honest strategy}

The algorithm behind agent decisions is called a strategy. We can define an "honesty factor" of the provider strategy as 
the average value of $E_{PN}(a)/(E_{PN} + E_{PG})$ for all agreements this provider participates in (and similarly for the requestor).
I.e. the more agent expects to gain against the agreement, the more dishonest they are.

Note: agents' intentions don't matter here - there's no difference if the agent breaks the agreement on purpose or accidentally.

Agent always tries to maximize their own utility (the "expected utility maximization" assumption), in other words: agent always selects the most profitable strategy.
Let's now paraphrase the third goal from the previous section as:

\begin{conclusion}

The final purpose of the reputation system on Golem is to make honest strategies more profitable than dishonest strategies.

\label{main purpose conclusion}
\end{conclusion}


\subsection{Defining reputation}

There are few different ways to reduce the profitability of dishonest strategies:

\begin{itemize}
\item{Make dishonest strategies not available at all, e.g. make debit note acceptance force payment.}
\item{Make dishonest strategies hard to implement, e.g. hide/obfuscate some important components of Golem.}
\item{Add some mechanics that directly penalize dishonesty, e.g. require deposits and confiscate them when dishonesty is "proven"}
\end{itemize}

Neither of them has anything to do with the "reputation":

\begin{conclusion}

"Reputation system" is an attempt to solve the problem defined in the conclusion \ref{main purpose conclusion} in the following way:

\begin{itemize}
    \item{Make some additional information available to the market participants}
    \item{This information can be used to estimate the "honesty factor" of an agent, and thus improves the accuracy of the total agreement value estimation}
    \item{The better the total agreement value estimation, the less profitable it is to trade with dishonest agents}
    \item{The less profitable it is to trade with dishonest agents, the fewer/worse agreements they have}
    \item{The fewer/worse agreements dishonest agents have, the less profitable are dishonest strategies}
\end{itemize}

\end{conclusion}

\end{document}
